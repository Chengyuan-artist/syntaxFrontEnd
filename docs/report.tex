\documentclass{ctexrep}
\usepackage[margin=25mm]{geometry}
\usepackage{amsmath}
\usepackage{fancyhdr}
\usepackage{cite}
%\usepackage[linesnumbered,boxed]{algorithm2e}
\usepackage{listings}
\usepackage{xcolor}
\usepackage{enumitem}

\ctexset { section = { format={\Large \bfseries } } }


\definecolor{codegreen}{rgb}{0,0.6,0}
\definecolor{codegray}{rgb}{0.5,0.5,0.5}
\definecolor{codepurple}{rgb}{0.58,0,0.82}
\definecolor{backcolour}{rgb}{0.95,0.95,0.92}


\lstdefinestyle{mystyle}{
    backgroundcolor=\color{backcolour},   
    commentstyle=\color{codegreen},
    keywordstyle=\color{magenta},
    numberstyle=\tiny\color{codegray},
    stringstyle=\color{codepurple},
    basicstyle=\ttfamily\footnotesize,
    breakatwhitespace=false,         
    breaklines=true,                 
    captionpos=b,                    
    keepspaces=true,                 
    numbers=left,                    
    numbersep=5pt,                  
    showspaces=false,                
    showstringspaces=false,
    showtabs=false,                  
    tabsize=2
}

\lstset{style=mystyle}


\pagestyle{fancy}
\chead{华中科技大学计算机科学与技术学院课程设计报告}



\begin{titlepage}
    \title{Syntax FrontEnd Design}
    \author{Chengyuan ZHANG}
    \date{March 2021}
\end{titlepage}


\begin{document}
\maketitle

\chapter*{任\;务\;书}
\addcontentsline{toc}{section}{任务书}
\section*{设计内容}

在计算机科学中,抽象语法树\textit{(abstract syntax tree或者缩写为AST)},是将源代码的语法结构的用树的形式表示,树上的每个结点都表示源程序代码中的一种语法成分。之所以说是“抽象”,是因为在抽象语法树中,忽略了源程序中语法成分的一些细节,突出了其主要语法特征。
抽象语法树\textit{(Abstract Syntax Tree ,AST)}作为程序的一种中间表示形式,在程序分析等诸多领域有广泛的应用.利用抽象语法树可以方便地实现多种源程序处理工具,比如源程序浏览器、智能编辑器、语言翻译器等。

在《高级语言源程序格式处理工具》这个题目中,首先需要采用形式化的方式,使用巴克斯\textit{(BNF)}范式定义高级语言的词法规则\textit{(字符组成单词的规则)}、语法规则\textit{(单词组成语句、程序等的规则)}。再利用形式语言自动机的的原理,对源程序的文件进行词法分析,识别出所有单词;使用编译技术中的递归下降语法分析法,分析源程序的语法结构,并生成抽象语法树,最后可由抽象语法树生成格式化的源程序。\cite{b1}\cite{b2}


\section*{设计要求}
\begin{enumerate}
    \item 语言定义
    
    选定C语言的一个子集,要求包含:

    \begin{enumerate}[label={(\arabic*)}]
        \item 基本数据类型的变量、常量,以及数组。不包含指针、结构,枚举等。
        \item 双目算术运算符(+\;-\;*\;/\;\%),关系运算符、逻辑与、逻辑或、赋值运算符。不包含逗号运算符、位运算符、各种单目运算符等等。
        \item 函数定义、声明与调用。
        \item 表达式语句、复合语句、if语句的2种形式、while语句、for语句,return语句、break语句、continue语句、外部变量说明语句、局部变量说明语句。
        \item 编译预处理(宏定义,文件包含)
        \item 注释(块注释与行注释)
    \end{enumerate}
    \item 单词识别
    
    设计DFA的状态转换图(参见实验指导),实验时给出DFA,并解释如何在状态迁移中完成单词识别(每个单词都有一个种类编号和单词的字符串这2个特征值),最终生成单词识别(词法分析)子程序。

    注:含后缀常量,以类型不同作为划分标准种类编码值,例如123类型为int,123L类型为long,单词识别时,种类编码应该不同;但0x123和123类型都是int,种类编码应该相同。
    \item 语法结构分析
    \begin{enumerate}[label={(\arabic*)}]
        \item 外部变量的声明;
        \item 函数声明与定义;
        \item 局部变量的声明;
        \item 语句及表达式;
        \item 生成(1)-(4)(包含编译预处理和注释)的抽象语法树并显示。
    \end{enumerate}
    \item 按缩进编排生成源程序文件。
    \item 评测说明
 
    要求具有如下功能:
    \begin{enumerate}[label={(\arabic*)}]
        \item \textbf{识别语言的全部单词(50\%):}
        
        要求测试用例包含所有种类的单词,测试用例中没有出现的单词种类视作没有完成该类单词的识别。由于每类单词有一个种类编码(参见实验指导书用枚举常量定义),可以将识别出来的单词按种类编码进行排序显示,这样既能方便自己的调试,也能方便检查。注意相同种类编码的多种形式,都应该包含在测试用例中,例如类型为int的常量,有三种形式0123、123、0x123。
        
        报错功能,指出不符合单词定义的符号位置。

        \item \textbf{语法结构分析与生成抽象语法树(40\%):}
        
        要求测试用例包含函数声明,定义、表达式(各种运算符均在某个表达式中出现)、所有的语句,以及if语句的嵌套,循环语句的嵌套。测试用例中没有出现的语句和嵌套结构,视作没有完成该种语法结构的分析。
        
        报错功能,指出不符合语法规则的错误位置。测试文件中不必包含错误语句等,检查时由老师随机修改测试文件,设置错误,检查报错功能是否实现。

        显示抽象语法树,要求能由抽象语法树说明源程序的语法结构,这也是检查时验证语法结构分析正确性的依据。
        \item \textbf{缩进编排重新生成源程序文件(10\%):}
        
        对(2)的测试用例生成的抽象语法树进行先根遍历,按缩进编排的方式写到.c文件中,查看文件验证是否满足任务要求。

    \end{enumerate}

\end{enumerate}




\bibliographystyle{plain}

\bibliography{TaskPageRef}    


\tableofcontents


\chapter{引言}
\section{背景与研究意义}

\section{国内外研究现状}
\section{课程设计的主要研究工作}


\chapter{系统需求分析与总体设计}
\section{系统需求分析}
\section{系统总体设计}


\begin{lstlisting}[language=C]
    #include<stdio.h>
    int main(){
    }
\end{lstlisting}


\chapter{系统详细设计}
\section{有关数据结构的定义}
\section{主要算法设计}


\chapter{系统实现与测试}
\section{系统实现}
\section{系统测试}


\chapter{结论}
\section{讨论与总结}
\section{未来展望}

\chapter{心得体会}



  


\end{document}